%--------------------
% Packages
% -------------------
\documentclass[11pt,a4paper]{article}
\usepackage[utf8x]{inputenc}

\usepackage[pdftex]{graphicx} % Required for including pictures
\usepackage[pdftex,linkcolor=black,pdfborder={0 0 0}]{hyperref} % Format links for pdf
\usepackage{calc} % To reset the counter in the document after title page
\usepackage{enumitem} % Includes lists

\usepackage{ amssymb } % extra math symbols
\usepackage{ amsmath } % extra math symbols
\usepackage{ dsfont } % font za množice
% tabele
\usepackage{array}
\usepackage{wrapfig}
\usepackage{multirow}
\usepackage{tabularx}

\frenchspacing % No double spacing between sentences
\setlength{\parindent}{0pt}
\setlength{\parskip}{0.1em}

\usepackage[a4paper, lmargin=0.1666\paperwidth, rmargin=0.1666\paperwidth, tmargin=0.1111\paperheight, bmargin=0.1111\paperheight]{geometry} %margins

\usepackage{lipsum} % Used for inserting dummy 'Lorem ipsum' text into the template

\begin{document} 


\subsection*{Oznake}
\begin{itemize}
    \item $\mathbb{N} = \{0,1,...\}$ množica naravnih števil
    \item $\mathbb{Z} = \{...,-1,0,1,...\}$ množica celih števil
    \item $[n] = \{1,...,n\}$
    \item $Y^X=\{f:X\to Y\}$ množica vseh preslikav iz $X$ v $Y$
    \item $ 2^X = P(X) = \{A \subseteq X\}$ množica vseh podmnožic množice $X$
    \item $ \delta_{i\,j} = \begin{cases}
        1;\ j=i\\
        0;\ j\neq i\\
    \end{cases}$ Kroneckerjeva delta 
\end{itemize}


\subsection*{Lastnosti funkcji}
\begin{itemize}
    \item \textbf{Injektivnost} (različna elementa se vedno slikata v različno sliko)
    \[\forall x,x' \in X : x\neq x' \Rightarrow f(x) \neq f(x')\]
    \[f(x) = f(x') \Rightarrow x = x'\]
    \[\exists f: X \to Y\ \textmd{injekcija} \Rightarrow |X| \leq |Y|\]
    \item \textbf{Surjektivnost} (vsak element $Z_f$ je slika vsaj enega elementa $D_f$)
    \[\forall y \in Y\ \exists x \in X: y = f(x)\]
    \[\exists f: X \to Y\ \textmd{surjekcija} \Rightarrow |X| \geq |Y|\]
    \item \textbf{Bijektivnost} (injekcija in surjekcija)
    \[\forall y \in Y\ \nexists x \in X : y = f(x)\]
    \[\exists f: X \to Y\ \textmd{bijekcija} \Rightarrow |X| = |Y|\]
\end{itemize}


\subsection*{Dirichletovo načelo}
Če $n$ kroglic razporedimo v $k$ škatel in je $n>k$, bosta v vsaj eni škatli vsaj dve kroglici.
\[\exists f: X \to Y\ \textmd{injekcija} \Rightarrow |X| \leq |Y|\]
\[|X| > |Y| \Rightarrow \nexists \textmd{injekcija}\ f: X \to Y \]

\subsection*{Posplošeno Dirichletovo načelo}
Če $n$ kroglic razporedimo v $k$ škatel in je $n>r \cdot k$, bo v vsaj eni škatli vsaj $r+1$ kroglic.

\subsection*{Načelo vsote}
\[ A\cap B = \emptyset \Rightarrow |A \cup B| = |A| + |B| \]

\subsection*{Načelo vklučitev in izključitev}
\[ |A \cup B| = |A| + |B| - |A \cap B| \]
\[ |A \cup B \cup C| = |A| + |B| + |C| - |A \cap B| - |A \cap C| - |B \cap C| + |A \cup B \cup C|\]

\subsection*{Načelo produkta}
\[ |A \times B| = |A| \cdot |B| \]

\subsection*{Asimptotična enakost}
\[a_n \sim b_n \Leftrightarrow \lim_{n \to \infty} \frac{a_n}{b_n} = 1\]

\subsection*{Eulerjeva funkcija}
\[ \varphi (n) = |\{k\in [n] : D(n,k) = 1 \}| = \textmd{št. proti $n$ tujih števil, ki so $\leq$ $n$}\]
\[\varphi(p) = p-1 \qquad p \in \mathbb{P}\]
\[ \varphi(p^k) = p^k-p^{k-1} = p^k(1-\frac{1}{p})\]
\[\sum_{d|n} \varphi(d) = n\]

\subsection*{Funkcije in urejene izbire}
$N$ in $K$ sta množici, kjer je $n = |N|$ in $k = |K|$. $K^N$ je množica vseh funkcij, ki slikajo iz $N$ v $K$.

(\textbf{Variacije s ponavljanjem}) Število poljubnih funkcij, ki slikajo iz $N$ v $K$ je $k^n$.
(\textbf{Variacije brez ponavljanja}) Število injektivnih funkcij, ki slikajo iz $N$ v $K$ je $n^{\underline{k}} = \frac{k!}{(k-n!)} $.

\subsubsection*{Padajoča potenca}
\[k^{\underline{n}} = (k)_n = k \cdot (k-1) \cdot (k-2) \cdot \dots \cdot (k - n + 1) = \frac{k!}{(k-n)!}\]

\subsubsection*{Naraščujoča potenca}
\[k^{\overline{n}} = (k)^n = k \cdot (k+1) \cdot (k+2) \cdot \dots \cdot (k + n - 1) = \frac{(k+n-1)!}{(k-1)!}\]

\subsubsection*{Stirlingova formula}
\[n! \sim \sqrt{2 \pi n} (\frac{n}{e})^n\]

\subsubsection*{Funkcija gama}
Funkcija gama je posplošitev fakultete.
\[x>0 \qquad \Gamma(x) = \int_0^{\infty} t^{x-1}e^{-t}dt\]
\[\Gamma(n+1) = n! \qquad \Gamma(\frac{1}{2}) = \sqrt{\pi}\]

\section*{Podmnožice in načrti}
\subsubsection*{Binomski koeficient}
Za nožico $N$ in število $k$ je \emph{binomski koeficient} množica vseh podmnožic množice $N$ moči $k$:
\[ \binom{N}{k} = \{A \subseteq N : |A| = k \} \]
Za $n,k \in \mathbb{N}$ je \emph{binomski koeficient}:
\[ \binom{n}{k} = \left|\binom{[n]}{k}\right|\]
Za binomski koeficient veljajo naslednje enaksti:
\begin{align*}
    \binom{n}{0}&=1 & \binom{n}{1}&=n & \binom{n}{n}&=1
\end{align*}
\begin{align*}
    \binom{n}{k}&=0 \quad \textmd{za $k > n$} & \binom{n}{k}&=\binom{n}{n-k}
\end{align*}
\[\binom{n}{k} = \frac{n^{\underline{k}}}{k!} = \frac{n(n-1)...(n-k+1)}{k!} = \frac{n!}{k!(n-k)!}\]
Obstaja tudi \emph{rekurzivna zveza}:
\[\binom{n}{k} = \binom{n-1}{k-1} + \binom{n-1}{k}\]

\subsubsection*{Binomski izrek}
\[(a+b)^n = \sum_{k=0}^n \binom{n}{k} a^{n-k} b^k\]

\subsection*{Izbori}
Imamo $n$ optevilčenih kroglic. Na koliko načinov lahko izberemo $k$ kroglic?

\begin{center}
    \begin{tabular}{ m{6em} | c | c | } 
         & \textbf{s ponavljanjem} & \textbf{brez ponavljanja}\\ 
        \hline
        \textbf{variacije} \emph{vrstni red je pomemben} & $n^k$ & $n^{\underline{k}}$ \\ 
        \hline
        \textbf{kombinacije} \emph{vrstni red ni pomemben} & $\binom{n+k-1}{k}$ & $\binom{n}{k}$ \\ 
    \end{tabular}
\end{center}

\subsection*{Kompozicije}
Kompozicija števila $n$ je $l$-terica števil $\lambda_i$, katerih vsota je $n$.
\[\lambda = (\lambda_1, ..., \lambda_l); \qquad \lambda_i \in \mathbb{Z}, \ \lambda_i > 0\]
\begin{align*}
    &\lambda_1, ..., \lambda_l\                 &   \ldots & \ &\textmd{členi kompozicije} \\
    &l(\lambda) = l\                            &   \ldots & \ &\textmd{dolžina kompozicije} \\
    &|\lambda| = \lambda_1 + ... + \lambda_l\   &   \ldots & \ &\textmd{velikost kompozicije} \\
\end{align*}

Število $n$ ima $2^{n-1}$ kompozicij in $\binom{n-1}{k-1} = \binom{n-1}{n-k}$ kompozicij s $k$ členi. 

\subsection*{Šibke kompozicije}
Šibke kompozicije se od navadnih razlikujejo po tem, da lahko vsebujejo tudi $0$.
\[\lambda = (\lambda_1, ..., \lambda_l); \qquad \lambda_i \in \mathbb{Z}, \ \lambda_i \geq 0\]

Število $n$ ima $\infty$ šibkih kompozicij in $\binom{n+k-1}{n} = \binom{n+k-1}{k-1}$ šibkih kompozicij s $k$ členi. 

\subsection*{Načelo vključitev in izključitev}
\[|A \cup B| = |A| + |B| - |A \cap B| \]
\[|A \cup B \cup C| = |A| + |B| + |C| - |A \cap B| - |A \cap C| - |B \cap C| - |A \cap B \cap C| \]

\begin{align*}
    |A_1 \cup ... \cup A_n| & = |A_1| + ... + |A_n| \\
        & - |A_1 \cap A_2| - |A_1 \cap A_3| - ... - |A_1 \cap A_n| - ... - |A_{n-1} \cap A_n| \\
        & + |A_1 \cap A_2 \cap A_3| + |A_1 \cap A_2 \cap A_4| + ... + |A_{n-2} \cap A_{n-1} \cap A_n| \\
        & - |A_1 \cap A_2 \cap A_3 \cap A_4| + ... \\
        & = \sum_{i=1}^n (-1)^{i-1} \sum_{1 \leq j_1 \le ... \le j_i \leq n} |A_{j_1} \cap A_{j_2} \cap ... \cap A_{j_i} | \\
    \left|\bigcup_{i=1}^n A_i\right| & = \sum_{\emptyset \neq S \subseteq [n]} (-1)^{|S| - 1} \left|A_S \right| \, ; \qquad A_S = \bigcap_{i\in S } A_i \\
    \left|\bigcap_{i=1}^n A_i^C\right| & = \sum_{S \subseteq [n]} (-1)^{|S|} \left| A_S \right|
\end{align*}

\subsection*{Načrti}
$\mathcal{B}=\{B_1,B_2,...,B_b\}$ je načrt s parametri $(v,k,\lambda)$, če velja:
\begin{itemize}
    \item $B_1,..., B_b \subseteq X$
    \item $|X| = v$
    \item $|B_1| = ... = |B_b| = k$
    \item Za $\forall i \in X$ je $i\in B_j$ za $\lambda$ različnih $j$-jev
\end{itemize}
Načrt lahko prikažemo s tabelo kljukic v kateri stolpci predstavljajo bloke vrstice pa elemente množice $X$.
\begin{center}
    \begin{tabular}{  c | c } 
         & $B \in \mathcal{B}$ \\ 
        \hline
        $x \in X$ & $\begin{cases}
            \checkmark\,; x\in B\\
            0\,; x \notin B
        \end{cases}$\\
    \end{tabular}
\end{center}
V vsakem stolpcu je $k$ kljukic.\\
V vsaki vrstici je $\lambda$ kljukic.

\[b\cdot k = v\cdot \lambda \Rightarrow k\, |\, v \cdot \lambda \]
\begin{align*}
    b &\leq \binom{v}{k} & \frac{v\cdot \lambda}{k} \leq \binom{v}{k}
\end{align*}
\[\lambda \leq \frac{k}{v}\frac{v!}{k!(v-k)!} = \frac{(v-1)!}{(k-1)!(v-k)!} = \binom{v-1}{k-1}\]
\\

Načrt s pareametri $(v,K,\lambda)$ obstaja $\Leftrightarrow$
\[k\,|\,v\cdot \lambda\ , \qquad \lambda \leq \binom{v-1}{k-1}\]

\subsection*{$t$-načrti}

$\mathcal{B}$ je $t$-načrt s parametri $(v,k,\lambda_t)$, če 
\begin{itemize}
    \item $B_i \subseteq X$
    \item $|X| = v$
    \item $|B_i| = k$
    \item $\forall S \subseteq X, |S| = t$ velja $S\subseteq B_i$ za natanko $\lambda_t$ indeksov $i$.
\end{itemize}

Če je $\mathcal{B}$ $t$-načrt s parametri $(v,k,\lambda_t)$, 
je tudi $(t-1)$-načrt s parametri $(v,k,\lambda_{t-1})$
kjer je
\[\lambda_{t-1} = \lambda_{t}\cdot \frac{v-t+1}{k-t+1}\]

Če je $\mathcal{B}$ $t$-načrt s parametri $(v,K,\lambda_t)$, potem je $\mathcal{B}$ tudi $s$-načrt s parametri $(v,K,\lambda_s)$ kjre je $1\leq s \leq t$ in
\[\lambda_s = \lambda_t\cdot \frac{v-t+1}{k-t+1}\cdot\frac{v-t+2}{k-t+2}\cdot...\cdot\frac{v-s}{k-s}\]

\section*{Permutacije, razdelitve in razčlenitve}
\subsection*{Stirlingova števila 1. vrste}
Permutacijo lahko zapišemo kot produkt disjunktnih ciklov.

$c(n,k)$ $\dots$ število premutacij v $S_n$ s $k$ cikli
\begin{align*}
    c(n,n) &= 1 & c(n,n-1) &= \binom{n}{2} & c(n,1) &= (n-1)! & c(n,0) &= \delta_{n\,0} 
\end{align*}

Za \emph{Stirlingova števila 1. vrste} ni enostavne formule imamo pa rekurzivno zvezo:
\[c(n,k) = c(n-1,k-1) + (n-1)\, c(n-1,k)\]

Vseh premutacij v $S_n$ je
\[\sum_k c(n,k) = n!\]
Izrek:
\[\sum_k c(n,k)x^k = x^{\overline{n}} \]

\subsection*{Stirlingova števila 2. vrste}
$\mathcal{B} = {B_1, ..., B_k}$ je razdelitev \textit{(razbitje, praticija)} množice $X$, če velja:
\begin{itemize}
    \item $B_i \neq \emptyset$
    \item $B_i \cap B_j = \emptyset$
    \item $\bigcup_{i=1}^k B_i = X$
\end{itemize}


$S(n,k)$ $\dots$ število razdelitev $[n]$ s $k$ bloki. 
\begin{align*}
    S(n,n) &= 1 & S(n,n-1) &= \binom{n}{2} & S(n,1) &= 1 & S(n,0) &= \delta_{n\,0}
\end{align*}
\[S(n,k) = 0 \quad \text{za $k>n$ ali $k<0$}\]

Število surjekcij iz $[n]$ v $[k]$ je enako $k! S(n,k)$.
Po načelu vključitev in izključitev je število surjekcij iz $[n]$ v $[k]$ enako:
\[\sum_{j=0}^k (-1)^{k-j} \binom{k}{j} j^n\]

Torej je
\[S(n,k)=\frac{\sum_{j=0}^n (-1)^{k-j}\binom{k}{j}j^n}{k!} \]

Za \emph{Stirlingova števila 2. vrste} velja tudi rekurzivna zveza:
\[S(n,k) = S(n-1,k-1) + k\,S(n-1,k)\]

Izrek:
\[ \sum_k S(n,k)x^{\underline{k}} = x^n\]

$S(n,k)$ je enako številu ekvivalenčnih relacij z $k$ ekvivalenčnimi razredi na $[n]$

\subsection*{Bellova števila}
$B(n)$ $\dots$ število vseh razdelitve $[n]$
\[B(n) = \sum_k S(n,k)\]
Rekurzivna zveza:
\[B(n+1) = \sum_{k=0}^n \binom{n}{k} B(k)\]

$B(n)$ je enako številu ekvivalenčnih relacij na $[n]$

\subsection*{Lahova števila}
$L(n,k)$ $\dots$ število razdelitev $[n]$ na $k$ med seboj linearno urejenih blokov
Rekurzivna zveza:
\[L(n,k) = L(n-1,k-1) + (n-1+k)\,L(n-1,k)\]
Izrek:
\[\sum_k L(n,k)x^{\underline{k}} = x^{\overline{n}}\]
Formula za \emph{Lahova števila}:
\[L(n,k) = \frac{n!}{k!}\binom{n-1}{k-1}\]

\subsection*{Multinomski koeficient}
\[\binom{n}{n_1, ..., n_k} = \frac{n!}{n_1!\cdot ... \cdot n_k!}\]
$n_1 + ... n_k$ mora biti enako $n$

\subsubsection*{Multinomski izrek}
\[(x_1 + ... + x_k)^n = \sum_{\underbrace{(n_1, ..., n_k)}_{\text{šibka kompozicija $n$}}} \binom{n}{n_1, ..., n_k} x_1^{n_1} x_2^{n_2} ...\, x_k^{n_k} \]

\subsection*{Razčlenitev}
$\lambda = (\lambda_1, ..., \lambda_l),\ \lambda_1 \geq ... \geq \lambda_l > 0,\ \lambda_i \in \mathbb{N}$ je razčlenitev števila $n = \lambda_1 + ... \lambda_l$

\begin{itemize}
    \item $\lambda_1, ...m \lambda_l$ so \emph{členi}
    \item $\lambda_1 + ... + \lambda_l = |\lambda|\ $ je velikost $\lambda$ 
    \item $l = l(\lambda)\ $ dolžina
\end{itemize}

Razčlenitev lahko grafično predstavimo z \textbf{Ferrersovim diagramom}:\\
V $i$. vrstico narišemo $\lambda_i$ pikic.
Naprimer $(4,3,3,1,1,1)$ je razčlenitev $13$.
\begin{align*}
    \begin{matrix}
        \bullet & \bullet & \bullet & \bullet \\
        \bullet & \bullet & \bullet  \\
        \bullet & \bullet & \bullet  \\
        \bullet \\
        \bullet \\
        \bullet \\
    \end{matrix}
\end{align*}

\textbf{Konjugirana razčlenitev} $\lambda'$ ali $\lambda^C$

Dobimo jo tako, da diagram transponiramo.\\
Naprimer $433111' = 6331$:
\begin{align*}
    \begin{matrix}
        \bullet & \bullet & \bullet & \bullet & \bullet & \bullet \\
        \bullet & \bullet & \bullet  \\
        \bullet & \bullet & \bullet  \\
        \bullet \\
    \end{matrix}
\end{align*}

\[\lambda_i' = |\{j:\lambda_j \geq i\}| = \max\{j: \lambda_j \geq i\}\]

\begin{align*}
    |\lambda'| &= |\lambda| & l(\lambda') &= \lambda_1 & \lambda_1' &= l(\lambda) & \lambda'' = \lambda
\end{align*}

$p(n)$ $\dots$ število razčlenitev $n$\\
$p_k(n)$ $\dots$ število razčlenitev $n$ s $k$ členi\\
$\overline{p_k}(n)$ $\dots$ število razčlenitev $n$ z največ $k$ členi\\

Rekurzivne zveze:
\[p_k(n) = \overline{p_n}(n-k)\]
\[p_k(n) = p_{n-1}(n-1) + p_k(n-k)\]
\[\overline{p_k}(n) = p_k(n) + \overline{p_{k-1}}(n) = \overline{p_{k-1}}(n) + \overline{p_k}(n-k)\]

Eulerjev petkotniški izrek
\[p(n) = \sum_{k=1}^{\infty} (-1)^{k-1} \left( p \left( n - \frac{k(3k-1)}{2}\right) + p \left( n - \frac{k(3k+1)}{2}\right)\right)\]


\subsection*{Dvanajstera pot}
Imamo $n$ kroglic in $k$ škatel. Na koliko načinov lahko damo kroglice v škatle. To je analogija za preslikave.
\begin{center}
\begin{tabular}{|l|l||l|l|l|}
    \hline
kroglice    & škatle    & \textbf{vse preslikave} & \textbf{injekcije}                              & \textbf{surjekcije} \\ \hline
ločimo      & ločimo    & $k^n$                 & $k^{\underline{n}}$                               & $k!\,S(n,k)$ \\ \hline
ne ločimo   & ločimo    & $\binom{n+k-1}{k-1}$  & $\binom{k}{n}$                                    & $\binom{n-1}{k-1}$ \\ \hline
ločimo      & ne ločimo & $\sum_{i=0}^k S(n,i)$ & $\begin{cases}1;\ k \geq n\\ 0; k < n\end{cases}$ & $S(n,k)$ \\ \hline
ne ločimo   & ne ločimo & $\overline{p_k}(n)$   & $\begin{cases}1;\ k \geq n\\ 0; k < n\end{cases}$ & $p_k(n)$ \\ \hline
\end{tabular}
\end{center}

\section*{Rodovne funkcije}
Kako lahko predstavimo zaporedje:
\begin{enumerate}
    \item \textbf{Z eksplicitno formulo:}
    \begin{align*}
        a_n &= 2^n & b_n &= n! & F_n &= \frac{1}{\sqrt{5}} \left( \left( \frac{1+\sqrt{5}}{2} \right)^{n+1} - \left( \frac{1-\sqrt{5}}{2} \right)^{n+1}  \right)
    \end{align*}
    \item \textbf{Z rekurzivno formulo:}
    \[ a_n = F(a_0, a_1, ..., a_{n-1}, n) \qquad \text{in začetni členi } a_i\]
    \begin{align*}
        a_n &= 2a_{n-1} & b_n &= n(b_{n-1}) & F_n &= F_{n-1} + F_{n-2}\\
        a_0 &= 1        &   b_0 &= 1        &   F_0 &= F_1 = 1 
    \end{align*}
    \item \textbf{S približkom oziroma asimptotično formulo}
    \begin{align*}
        b_n &\sim \sqrt{2\pi n} \left( \frac{n}{e} \right)^n & F_n &\sim \frac{1}{\sqrt{5}} \left( \frac{1+\sqrt{5}}{2}\right)^{n+1} \\
    \end{align*}
    \[a_n \sim b_n \Leftrightarrow \lim_{n \to \infty} \frac{a_n}{b_n} = 1\]
    \item \textbf{Z rodovno funkcijo}
    
\end{enumerate}

\end{document}