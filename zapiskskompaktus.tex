\documentclass[a4paper,8pt]{extarticle}
\usepackage{amssymb,amsmath,amsthm,amsfonts}
\usepackage{multicol,multirow}
\usepackage{calc}
\usepackage{ifthen}
\usepackage{tabularx}
\usepackage[utf8]{inputenc}
\usepackage[landscape]{geometry}
\usepackage[colorlinks=true,citecolor=blue,linkcolor=blue]{hyperref}
\usepackage{accents}
\newcommand{\vect}[1]{\accentset{\rightharpoonup}{#1}}

\ifthenelse{\lengthtest { \paperwidth = 11in}}
    { \geometry{top=.5in,left=.5in,right=.5in,bottom=.5in} }
	{\ifthenelse{ \lengthtest{ \paperwidth = 297mm}}
		{\geometry{top=1cm,left=1cm,right=1cm,bottom=1cm} }
		{\geometry{top=1cm,left=1cm,right=1cm,bottom=1cm} }
	}
\pagestyle{empty}
\makeatletter
\renewcommand{\section}{\@startsection{section}{1}{0mm}%
                                {-1ex plus -.5ex minus -.2ex}%
                                {0.5ex plus .2ex}%x
                                {\normalfont\large\bfseries}}
\renewcommand{\subsection}{\@startsection{subsection}{2}{0mm}%
                                {-1explus -.5ex minus -.2ex}%
                                {0.5ex plus .2ex}%
                                {\normalfont\normalsize\bfseries}}
\renewcommand{\subsubsection}{\@startsection{subsubsection}{3}{0mm}%
                                {-1ex plus -.5ex minus -.2ex}%
                                {1ex plus .2ex}%
                                {\normalfont\small\bfseries}}
\makeatother
\setcounter{secnumdepth}{0}
\setlength{\parindent}{0pt}
\setlength{\parskip}{0pt plus 0.5ex}
% -----------------------------------------------------------------------

\title{Kombinatorika}

\begin{document}

\raggedright
\footnotesize

\begin{multicols}{4}
\setlength{\premulticols}{1pt}
\setlength{\postmulticols}{1pt}
\setlength{\multicolsep}{1pt}
\setlength{\columnsep}{2pt}






\subsection*{Oznake}
\begin{itemize}
    \item $\mathbb{N} = \{0,1,...\}$ množica naravnih števil
    \item $\mathbb{Z} = \{...,-1,0,1,...\}$ množica celih števil
    \item $[n] = \{1,...,n\}$
    \item $Y^X=\{f:X\to Y\}$ množica vseh preslikav iz $X$ v $Y$
    \item $ 2^X = P(X) = \{A \subseteq X\}$ množica vseh podmnožic množice $X$
    \item $ \delta_{i\,j} = \begin{cases}
        1;\ j=i\\
        0;\ j\neq i\\
    \end{cases}$ Kroneckerjeva delta 
\end{itemize}


\subsection*{Lastnosti funkcji}
\begin{itemize}
    \item \textbf{Injektivnost} (različna elementa se vedno slikata v različno sliko)
    \[\forall x,x' \in X : x\neq x' \Rightarrow f(x) \neq f(x')\]
    \[f(x) = f(x') \Rightarrow x = x'\]
    \[\exists f: X \to Y\ \textmd{injekcija} \Rightarrow |X| \leq |Y|\]
    \item \textbf{Surjektivnost} (vsak element $Z_f$ je slika vsaj enega elementa $D_f$)
    \[\forall y \in Y\ \exists x \in X: y = f(x)\]
    \[\exists f: X \to Y\ \textmd{surjekcija} \Rightarrow |X| \geq |Y|\]
    \item \textbf{Bijektivnost} (injekcija in surjekcija)
    \[\forall y \in Y\ \nexists x \in X : y = f(x)\]
    \[\exists f: X \to Y\ \textmd{bijekcija} \Rightarrow |X| = |Y|\]
\end{itemize}


\subsection*{Dirichletovo načelo}
Če $n$ kroglic razporedimo v $k$ škatel in je $n>k$, bosta v vsaj eni škatli vsaj dve kroglici.
\[\exists f: X \to Y\ \textmd{injekcija} \Rightarrow |X| \leq |Y|\]
\[|X| > |Y| \Rightarrow \nexists \textmd{injekcija}\ f: X \to Y \]

\subsection*{Posplošeno Dirichletovo načelo}
Če $n$ kroglic razporedimo v $k$ škatel in je $n>r \cdot k$, bo v vsaj eni škatli vsaj $r+1$ kroglic.

\subsection*{Načelo vsote}
\[ A\cap B = \emptyset \Rightarrow |A \cup B| = |A| + |B| \]

\subsection*{Načelo produkta}
\[ |A \times B| = |A| \cdot |B| \]

\subsection*{Asimptotična enakost}
\[a_n \sim b_n \Leftrightarrow \lim_{n \to \infty} \frac{a_n}{b_n} = 1\]

\subsection*{Eulerjeva funkcija}
\[ 
    \begin{aligned}
        \varphi (n) &= |\{k\in [n] : D(n,k) = 1 \}| \\
                &= \textmd{št. proti $n$ tujih števil, ki so $\leq$ $n$} \\
        \varphi(p) &= p-1 \qquad \qquad p \in \mathbb{P}\\
        \varphi(p^k) &= p^k-p^{k-1} = p^k(1-\frac{1}{p})\\
    \end{aligned}
\]
\[\sum_{d|n} \varphi(d) = n\]

\subsection*{Funkcije in urejene izbire}
$N$ in $K$ sta množici, kjer je $n = |N|$ in $k = |K|$. $K^N$ je množica vseh funkcij, ki slikajo iz $N$ v $K$.

\textbf{Variacije s ponavljanjem}

Število poljubnih funkcij, ki slikajo iz $N$ v $K$ je $k^n$.

\textbf{Variacije brez ponavljanja}

Število injektivnih funkcij, ki slikajo iz $N$ v $K$ je $n^{\underline{k}} = \frac{k!}{(k-n!)} $.

\subsubsection*{Padajoča potenca}
\[k^{\underline{n}} = (k)_n = k \cdot (k-1) \cdot (k-2) \cdot \dots \cdot (k - n + 1) = \frac{k!}{(k-n)!}\]

\subsubsection*{Naraščujoča potenca}
\[k^{\overline{n}} = (k)^n = k \cdot (k+1) \cdot (k+2) \cdot \dots \cdot (k + n - 1) = \frac{(k+n-1)!}{(k-1)!}\]

\subsubsection*{Stirlingova formula}
\[n! \sim \sqrt{2 \pi n} (\frac{n}{e})^n\]

\subsubsection*{Funkcija gama}
Funkcija gama je posplošitev fakultete.
\[x>0 \qquad \Gamma(x) = \int_0^{\infty} t^{x-1}e^{-t}dt\]
\[\Gamma(n+1) = n! \qquad \Gamma(\frac{1}{2}) = \sqrt{\pi}\]

\section*{Podmnožice in načrti}
\subsubsection*{Binomski koeficient}
Za nožico $N$ in število $k$ je \emph{binomski koeficient} množica vseh podmnožic množice $N$ moči $k$:
\[ \binom{N}{k} = \{A \subseteq N : |A| = k \} \]
Za $n,k \in \mathbb{N}$ je \emph{binomski koeficient}:
\[ \binom{n}{k} = \left|\binom{[n]}{k}\right|\]
Za binomski koeficient veljajo naslednje enaksti:
\begin{align*}
    \binom{n}{0}&=1 & \binom{n}{1}&=n & \binom{n}{n}&=1
\end{align*}
\begin{align*}
    \binom{n}{k}&=0 \quad \textmd{za $k > n$} & \binom{n}{k}&=\binom{n}{n-k}
\end{align*}
\[\binom{n}{k} = \frac{n^{\underline{k}}}{k!} = \frac{n(n-1)...(n-k+1)}{k!} = \frac{n!}{k!(n-k)!}\]
Obstaja tudi \emph{rekurzivna zveza}:
\[\binom{n}{k} = \binom{n-1}{k-1} + \binom{n-1}{k}\]

\subsubsection*{Binomski izrek}
\[(a+b)^n = \sum_{k=0}^n \binom{n}{k} a^{n-k} b^k\]

\subsection*{Izbori}
Imamo $n$ optevilčenih kroglic. Na koliko načinov lahko izberemo $k$ kroglic?

\begin{center}
    \begin{tabular}{ m{6em} | c | c | } 
         & \textit{s ponavljanjem} & \textit{brez ponavljanja}\\ 
        \hline
        \textbf{variacije} \emph{vrstni red je pomemben} & $n^k$ & $n^{\underline{k}}$ \\ 
        \hline
        \textbf{kombinacije} \emph{vrstni red ni pomemben} & $\binom{n+k-1}{k}$ & $\binom{n}{k}$ \\ 
    \end{tabular}
\end{center}

\subsection*{Kompozicije}
Kompozicija števila $n$ je $l$-terica števil $\lambda_i$, katerih vsota je $n$.
\[\lambda = (\lambda_1, ..., \lambda_l); \qquad \lambda_i \in \mathbb{Z}, \ \lambda_i > 0\]
\begin{align*}
    &\lambda_1, ..., \lambda_l\                 &   \ldots & \ &\textmd{členi kompozicije} \\
    &l(\lambda) = l\                            &   \ldots & \ &\textmd{dolžina kompozicije} \\
    &|\lambda| = \lambda_1 + ... + \lambda_l\   &   \ldots & \ &\textmd{velikost kompozicije} \\
\end{align*}

Število $n$ ima $2^{n-1}$ kompozicij in $\binom{n-1}{k-1} = \binom{n-1}{n-k}$ kompozicij s $k$ členi. 

\subsection*{Šibke kompozicije}
Šibke kompozicije se od navadnih razlikujejo po tem, da lahko vsebujejo tudi $0$.
\[\lambda = (\lambda_1, ..., \lambda_l); \qquad \lambda_i \in \mathbb{Z}, \ \lambda_i \geq 0\]

Število $n$ ima $\infty$ šibkih kompozicij in $\binom{n+k-1}{n} = \binom{n+k-1}{k-1}$ šibkih kompozicij s $k$ členi. 

\subsection*{Načelo vključitev in izključitev}
\[|A \cup B| = |A| + |B| - |A \cap B| \]
\[
    \begin{aligned}
        |A \cup B \cup C| &= |A| + |B| + |C| - |A \cap B| - |A \cap C| - \\
        &- |B \cap C| + |A \cap B \cap C| 
    \end{aligned}
\]

\begin{align*}
    &|A_1 \cup ... \cup A_n|  = \\
    &=\sum_{i=1}^n (-1)^{i-1} \sum_{1 \leq j_1 \le ... \le j_i \leq n} |A_{j_1} \cap A_{j_2} \cap ... \cap A_{j_i} | \\
\end{align*}
\begin{align*}
    \left|\bigcup_{i=1}^n A_i\right| & = \sum_{\emptyset \neq S \subseteq [n]} (-1)^{|S| - 1} \left|A_S \right| \, ; \quad A_S = \bigcap_{i\in S } A_i \\
    \left|\bigcap_{i=1}^n A_i^C\right| & = \sum_{S \subseteq [n]} (-1)^{|S|} \left| A_S \right|
\end{align*}

\subsection*{Načrti}
$\mathcal{B}=\{B_1,B_2,...,B_b\}$ je načrt s parametri $(v,k,\lambda)$, če velja:
\begin{itemize}
    \item $B_1,..., B_b \subseteq X$
    \item $|X| = v$
    \item $|B_1| = ... = |B_b| = k$
    \item Za $\forall i \in X$ je $i\in B_j$ za $\lambda$ različnih $j$-jev
\end{itemize}
Načrt lahko prikažemo s tabelo kljukic v kateri stolpci predstavljajo bloke vrstice pa elemente množice $X$.
\begin{center}
    \begin{tabular}{  c | c } 
         & $B \in \mathcal{B}$ \\ 
        \hline
        $x \in X$ & $\begin{cases}
            \checkmark\,; x\in B\\
            0\,; x \notin B
        \end{cases}$\\
    \end{tabular}
\end{center}
V vsakem stolpcu je $k$ kljukic.\\
V vsaki vrstici je $\lambda$ kljukic.

\[b\cdot k = v\cdot \lambda \Rightarrow k\, |\, v \cdot \lambda \]
\begin{align*}
    b &\leq \binom{v}{k} & \frac{v\cdot \lambda}{k} \leq \binom{v}{k}
\end{align*}
\[\lambda \leq \frac{k}{v}\frac{v!}{k!(v-k)!} = \frac{(v-1)!}{(k-1)!(v-k)!} = \binom{v-1}{k-1}\]
\\

Načrt s pareametri $(v,K,\lambda)$ obstaja $\Leftrightarrow$
\[k\,|\,v\cdot \lambda\ , \qquad \lambda \leq \binom{v-1}{k-1}\]

\subsection*{$t$-načrti}

$\mathcal{B}$ je $t$-načrt s parametri $(v,k,\lambda_t)$, če 
\begin{itemize}
    \item $B_i \subseteq X$
    \item $|X| = v$
    \item $|B_i| = k$
    \item $\forall S \subseteq X, |S| = t$ velja $S\subseteq B_i$ za natanko $\lambda_t$ indeksov $i$.
\end{itemize}

Če je $\mathcal{B}$ $t$-načrt s parametri $(v,k,\lambda_t)$, 
je tudi $(t-1)$-načrt s parametri $(v,k,\lambda_{t-1})$
kjer je
\[\lambda_{t-1} = \lambda_{t}\cdot \frac{v-t+1}{k-t+1}\]

Če je $\mathcal{B}$ $t$-načrt s parametri $(v,K,\lambda_t)$, potem je $\mathcal{B}$ tudi $s$-načrt s parametri $(v,K,\lambda_s)$ kjre je $1\leq s \leq t$ in
\[\lambda_s = \lambda_t\cdot \frac{v-t+1}{k-t+1}\cdot\frac{v-t+2}{k-t+2}\cdot...\cdot\frac{v-s}{k-s}\]

\section*{Permutacije, razdelitve in razčlenitve}
\subsection*{Stirlingova števila 1. vrste}
Permutacijo lahko zapišemo kot produkt disjunktnih ciklov.

$c(n,k)$ $\dots$ število premutacij v $S_n$ s $k$ cikli
\begin{align*}
    c(n,n) &= 1 & c(n,n-1) &= \binom{n}{2} & c(n,1) &= (n-1)! & c(n,0) &= \delta_{n\,0} 
\end{align*}

Za \emph{Stirlingova števila 1. vrste} ni enostavne formule imamo pa rekurzivno zvezo:
\[c(n,k) = c(n-1,k-1) + (n-1)\, c(n-1,k)\]

Vseh premutacij v $S_n$ je
\[\sum_k c(n,k) = n!\]
Izrek:
\[\sum_k c(n,k)x^k = x^{\overline{n}} \]

\subsection*{Stirlingova števila 2. vrste}
$\mathcal{B} = {B_1, ..., B_k}$ je razdelitev \textit{(razbitje, praticija)} množice $X$, če velja:
\begin{itemize}
    \item $B_i \neq \emptyset$
    \item $B_i \cap B_j = \emptyset$
    \item $\bigcup_{i=1}^k B_i = X$
\end{itemize}


$S(n,k)$ $\dots$ število razdelitev $[n]$ s $k$ bloki. 
\begin{align*}
    S(n,n) &= 1 & S(n,n-1) &= \binom{n}{2} & S(n,1) &= 1 & S(n,0) &= \delta_{n\,0}
\end{align*}
\[S(n,k) = 0 \quad \text{za $k>n$ ali $k<0$}\]

Število surjekcij iz $[n]$ v $[k]$ je enako $k! S(n,k)$.
Po načelu vključitev in izključitev je število surjekcij iz $[n]$ v $[k]$ enako:
\[\sum_{j=0}^k (-1)^{k-j} \binom{k}{j} j^n\]

Torej je
\[S(n,k)=\frac{\sum_{j=0}^n (-1)^{k-j}\binom{k}{j}j^n}{k!} \]

Za \emph{Stirlingova števila 2. vrste} velja tudi rekurzivna zveza:
\[S(n,k) = S(n-1,k-1) + k\,S(n-1,k)\]

Izrek:
\[ \sum_k S(n,k)x^{\underline{k}} = x^n\]

$S(n,k)$ je enako številu ekvivalenčnih relacij z $k$ ekvivalenčnimi razredi na $[n]$

\subsection*{Bellova števila}
$B(n)$ $\dots$ število vseh razdelitve $[n]$
\[B(n) = \sum_k S(n,k)\]
Rekurzivna zveza:
\[B(n+1) = \sum_{k=0}^n \binom{n}{k} B(k)\]

$B(n)$ je enako številu ekvivalenčnih relacij na $[n]$

\subsection*{Lahova števila}
$L(n,k)$ $\dots$ število razdelitev $[n]$ na $k$ med seboj linearno urejenih blokov
Rekurzivna zveza:
\[L(n,k) = L(n-1,k-1) + (n-1+k)\,L(n-1,k)\]
Izrek:
\[\sum_k L(n,k)x^{\underline{k}} = x^{\overline{n}}\]
Formula za \emph{Lahova števila}:
\[L(n,k) = \frac{n!}{k!}\binom{n-1}{k-1}\]

\subsection*{Multinomski koeficient}
\[\binom{n}{n_1, ..., n_k} = \frac{n!}{n_1!\cdot ... \cdot n_k!}\]
$n_1 + ... n_k$ mora biti enako $n$

\subsubsection*{Multinomski izrek}
\[(x_1 + ... + x_k)^n = \sum_{\underbrace{(n_1, ..., n_k)}_{\text{š. komp. $n$}}} \binom{n}{n_1, ..., n_k} x_1^{n_1} ...\, x_k^{n_k} \]

\subsection*{Razčlenitev}
$\lambda = (\lambda_1, ..., \lambda_l),\ \lambda_1 \geq ... \geq \lambda_l > 0,\ \lambda_i \in \mathbb{N}$ je razčlenitev števila $n = \lambda_1 + ... \lambda_l$

\begin{itemize}
    \item $\lambda_1, ...m \lambda_l$ so \emph{členi}
    \item $\lambda_1 + ... + \lambda_l = |\lambda|\ $ je velikost $\lambda$ 
    \item $l = l(\lambda)\ $ dolžina
\end{itemize}

Razčlenitev lahko grafično predstavimo z \textbf{Ferrersovim diagramom}:
v $i$. vrstico narišemo $\lambda_i$ pikic.


\textbf{Konjugirano razčlenitev} $\lambda'$ ali $\lambda^C$
dobimo tako, da diagram transponiramo.\\
Naprimer $433111' = 6331$

\[\lambda_i' = |\{j:\lambda_j \geq i\}| = \max\{j: \lambda_j \geq i\}\]

\begin{align*}
    |\lambda'| &= |\lambda| & l(\lambda') &= \lambda_1 & \lambda_1' &= l(\lambda) & \lambda'' = \lambda
\end{align*}

$p(n)$ $\dots$ število razčlenitev $n$\\
$p_k(n)$ $\dots$ število razčlenitev $n$ s $k$ členi\\
$\overline{p_k}(n)$ $\dots$ število razčlenitev $n$ z največ $k$ členi\\

Rekurzivne zveze:
\[p_k(n) = \overline{p_n}(n-k)\]
\[p_k(n) = p_{n-1}(n-1) + p_k(n-k)\]
\[\overline{p_k}(n) = p_k(n) + \overline{p_{k-1}}(n) = \overline{p_{k-1}}(n) + \overline{p_k}(n-k)\]

Eulerjev petkotniški izrek
\[ p(n) = \sum_{k=1}^{\infty} \textstyle (-1)^{k-1} \left( p \left( n - \frac{k(3k-1)}{2}\right) + p \left( n - \frac{k(3k+1)}{2}\right)\right)\]


\subsection*{Dvanajstera pot}
Imamo $n$ kroglic in $k$ škatel. Na koliko načinov lahko damo kroglice v škatle. To je analogija za preslikave.
\begin{center}
\begin{tabular}{|l|l||l|l|l|}
    \hline
$\bullet$    & $\sqcup$    & \textbf{vse} & \textbf{injekcije}                              & \textbf{surjekcije} \\ \hline
L      & L    & $k^n$                 & $k^{\underline{n}}$                               & $k!\,S(n,k)$ \\ \hline
N   & L    & $\binom{n+k-1}{k-1}$  & $\binom{k}{n}$                                    & $\binom{n-1}{k-1}$ \\ \hline
L      & N & $\sum_{i=0}^k S(n,i)$ & $\begin{cases}1;\ k \geq n\\ 0; k < n\end{cases}$ & $S(n,k)$ \\ \hline
N   & N & $\overline{p_k}(n)$   & $\begin{cases}1;\ k \geq n\\ 0; k < n\end{cases}$ & $p_k(n)$ \\ \hline
\end{tabular}
\end{center}

\section*{Rodovne funkcije}
\[
    \begin{aligned}
        \sum_{n=0}^{\infty} q^n &= \frac{1}{1-q} &
        \sum_{n=0}^{b} q^n &= \frac{1-q^{b+1}}{1-q}
        \\
        \sum_{n=a}^{\infty} q^n &= \frac{q^{a}}{1-q} &
        \sum_{n=a}^{b} q^n &= \frac{q^a-q^{b+1}}{1-q}
    \end{aligned}
\]
\[
    a^n - b^n = (a-b)(a^{n-1} + a^{n-2}b + ... + ab^{n-2} + b^{n-1})  
\]
\[ \textstyle \frac{a_0 + ... + a_{k-1}x^{k-1}}{1-x^k} = a_0 + ... + a_{k-1}x^{k-1} + a_0^k + ... + a_{k-1}x^{2k-1} + ...\]
\[ (x+y)^n = \sum_{k=0}^{n} \binom{n}{k} x^{n-k}y^{k} \]
\[ \frac{1}{(1-x)^n} = \sum_{k=0}^{n} \binom{n+k-1}{k} x^{k} \]
\[ B_\lambda(x) = \sum_{n} \binom{\lambda}{n} x^{n} = (1+x)^\lambda; \qquad \binom{\lambda}{n} = \frac{\lambda^{\underline{n}}}{n!}\]

\subsection{Rekurzivne enačbe}
\[c+bx+ax^2 = c(1-y_1 x)(1-y_2 x)\]
\[y_{1,2} = \frac{1}{x_{1,2}} = \frac{2a}{-b \pm \sqrt{b^2 - 4ac}} = \frac{-b \pm \sqrt{b^2-4ac}}{2c}\]

\subsubsection{Metoda prekrivanja}
\[
    \frac{q(x)}{(x-a_1)...(x-a_k)} = \frac{A_1}{x-a_1}...\frac{A_k}{x-a_k}
\]
pomnožimo z $(x-a_i)$ in vstavimo $x \leftarrow a_i$.

\subsubsection{Hornerjev algoritem}
\[a_n x^n + ... + a_0 = 0\]
\begin{itemize}
    \item možne cele ničle: $\pm$delitelji $a_0$
    \item možne racionalne ničle: $\pm \frac{\text{delitelji }a_0}{\text{delitelji }a_n} = k$
\end{itemize}
\begin{center}
    \begin{tabular}{ l|l l l l}
            & $a_n$ & $a_{n-1}$ & $...$ & $a_0$ \\ \hline
        $k$ &       & $ka_n$    & $...$ & \\ \hline
            & $a_n$ & $ka_n - a_{n-1}$ & $...$ & ostanek\\
    \end{tabular}
\end{center}
    
\subsubsection{Reševanje rekurzivne enačbe}
\[c_d a_n + c_{d-1} a_{n-1} + ... + c_0 a_{n-d} = q(n)\lambda^n\]
\begin{itemize}
    \item \textbf{Rešimo homogeni del} ($q(n)\lambda^n = 0$)
    \[
    \begin{aligned}
        c_d \lambda^d + c_{d-1} \lambda^{d-1} + ... + c_0 \quad &...\textit{ karakteristični pl.}\\
        \lambda_1, ..., \lambda_k \quad &...\textit{ ničle polinoma}\\
        \alpha_1, ..., \alpha_k \quad &...\textit{ večkratnosti ničel}
    \end{aligned}  
    \]
    \[a_n^\text{homo} = \sum_{i=1}^{k} p_i(n)\lambda_i^n\]
    $p_i(n) \quad ...$ \textit{polinom z neznanimi koeficienti}
    $\deg(p_i) < \alpha_i$
    
    \item \textbf{Izračunamo partikularni del}\\
    \emph{(če je enačba homogena to točko izpustimo)}
    \[a_n^\text{part} = r(n) n^a\lambda^n\]
    $r(n) \quad ...$ \textit{polinom z neznanimi koeficienti}
    $\deg(r) \leq \deg(q)$
    \[
    a = \begin{cases}
        0; & \lambda \text{ ni ničla karakterističnega pl.}\\
        \text{večkratnost } \lambda; & \lambda \text{ je ničla}\\
    \end{cases}
    \]
    
    $a_n^\text{part}$ vstavimo v originalno enačbo namesto $a_n$, združimo po $n$ in izračunamo neznane keoficiente polinoma $r(n)$

    \item \textbf{Združimo homogeni in partikularni del}
    \[a_n = a_n^\text{homo} + a_n^\text{part}\]

    $a_n$ enačimo z podanimi začetnimi členi $a_0, ..., a_{d-1}$ in izračunamo neznane koeficiente polinomov $p_i(n)$.
\end{itemize}

\subsection{Catalanova števila}
Catalanova števila štejejo:
\begin{itemize}
    \item št. postavitev oklepajev v izrazu $x \circ x \circ ... \circ x$ z $n$ $\circ$.
    \item št. binarnih dreves
    \item triangulacije pravilnega $n+2$ kotnika
\end{itemize}
\[C_n = \frac{1}{n+1} \binom{2n}{n}; \qquad n \geq 0\]
Rekurzivna zveza:
\[C_0 = 1, C_1 = 1, C_2 = 2, C_3 = 5\]
\[C_{n+1} =  \sum_{k=0}^{n}C_kC_{n-k}\]
\[C_n =  \sum_{k=0}^{n-1}C_kC_{n-k-1}\]

\section{Polyeva teorija}
\subsubsection{Orbita elementa $x$}
\[Gx = \{gx: g\in G\}\]
$x \in X$, $X$ je množica elementov.\\
$G$ je grupa (permutacij) ki deluje na $X$.

\subsubsection{Stabilizator elementa $x$}
\[G_x = \{g \in G: gx = x\}\]

\subsubsection{Izrek o orbiti in stabilizatorju}
\[|G| = |G_x| \cdot |Gx| \]

\subsubsection{Ciklični indeks grupe}
$g\in G$ : $\alpha_i$ ... št. ciklov dolžine $i$\\
$n = |X|$ 
\[Z_G (t_1, ..., t_n) = \frac{1}{|G|} \sum_{g \in G} t_1^{\alpha_1(g)} \cdot ... \cdot t_n^{\alpha_n(g)}\]

\subsubsection{Burnsidova lema}
\[ |X/G| = \frac{1}{|G|} \sum_{g \in G} |X_g| \]
$X/G$ ... množica vseh orbit\\
$X_g = \{x\in X : gx = x\}$ ... množica negibnih točk

\subsection{Barvanje}
Naj bo $X$ množica točk (npr. v grafu), $G$ grupa (permutacij) ki deljue na $X$.\\
Naj bo $K$ množica barvanj točk iz $X$ z $k$ barvami.\\
Število različnih barvanj je:
\[ |K/G| = \frac{1}{|G|} \sum_{g \in G} |K_g| \]
$|K_g|$ ... št. negibnih točk (barvanj, ki jih g ne spremeni)
Če ni drugih pogojev, je
$|K_g| = k^{\alpha(g)}$, kjer je $\alpha(g)$ št. ciklov v $g$.\\
\emph{Vsi elementi v ciklu morejo biti iste barve.}  

\subsubsection{Ciklični indeks in barvanje}
Število barvanj (s $k$ barvami) množice $X$ na katero deluje grupa $G$ je
\[Z_G(k,...,k)\]

Če imamo določeno število posamezne barve $m_1, ..., m_k$, je število barvanj koeficient pri:
\[[u_1^{m_1}...u_k^{m_k}] Z_G(u_1+...+u_k,u_1^2+...+u_k^2, ..., u_1^n+...+u_k^n)\]

\section{Teorija delno urejenih množic}
$(P,\leq)$ je delno urejena množica če velja:
\begin{itemize}
    \item refleksivnost: $x \leq x$
    \item antisimetričnost: $x \leq y \wedge y \leq x \Rightarrow x = y$
    \item tranzitivnost: $x \leq y \wedge y \leq z \Rightarrow x \leq z$
\end{itemize}

$x < y \Leftrightarrow x \leq y \wedge x \neq y$\\
$x \lessdot y \Leftrightarrow x < y \wedge \nexists z : x < z < y$\\

$x$ je največji element $\Leftrightarrow$ $\forall y \in P : y \leq x$\\
$x$ je maksimalen element $\Leftrightarrow$ $\forall y \in P : x \nless y$\\

\subsubsection{Hassejev diagram $(P,\leq)$}
 \[ \text{graf}(V,E), \quad V = P\]
 \[ x \sim y \Leftrightarrow x \lessdot y \text{ ali } y \lessdot x\]
Ponavadi večje elemente pišemo višje.
\subsubsection{Izomorfnost delno urejenih množic}
Delno urejeni množici $P$ in $Q$ sta izomorfni (oznaka: $\approxeq$), če obstaja preslikava $\varphi : P \to Q $ za katero velja:
\begin{itemize}
    \item bijektivnost
    \item $ x \leq_P y \Leftrightarrow \varphi(x) \leq_Q \varphi(y)$
\end{itemize}

\subsubsection{Kartezični produkt}
$P \times Q = \{ (x,y): x \in P, y \in Q\}$\\
$(x,y) \leq (x',y') \Leftrightarrow x \leq_P x' \wedge y \leq_Q y'$\\
$(P \times Q, \leq)$ je delno urejena množica

\subsubsection{Veriga}
Veriga v DUM $(P,\leq)$ je množica $C \subseteq P$ za katero velja:
\[\forall x,y \in C : x \leq y \vee y \leq x\]
\emph{Vsaka dva elemanta iz $C$ sta primerljiva}
\textbf{Višina} DUM je dolžina najdaljše verige.

\subsubsection{Antiveriga}
Antiveriga v DUM $(P,\leq)$ je množica $A \subseteq P$ za katero velja:
\[\forall x,y \in A : (x \nleq y \wedge y \nleq x) \vee x = y\]
\emph{Vsaka dva elemanta iz $A$ sta neprimerljiva}
\textbf{Širina} DUM je dolžina najdaljše antiverige.

\subsubsection{Minskyjev izrek}
dolžina najdaljše verige $=$ najmanjše št. antiverig, s katerimi lahko pokrijemo $P$

\subsubsection{Dilworhow izrek}
dolžina najdaljiše antiverige $=$ najmanjše št. verig s katerim lahko pokrijemo $P$

\subsubsection{Spernerjev izrek}
Širina $B_n$ je $\binom{n}{\left\lfloor\frac{n}{2}\right\rfloor}$.

$B_n = (2^{[n]}, \subseteq)$

Hassejev diagram $B_n$ je hiperkocka $Q_n$.

\subsubsection{Hallov izrek}
$G = (V,E)$ graf\\
\begin{itemize}
    \item $M \subseteq E$ je \textbf{prirejanje}, če velja $\forall e,f \in M : e \cap f = \emptyset$ \\
    (Če povezave $M$ pobarvamo rdeče, nobeno vozljišče nima več kot eno rdečo povezavo)\\
    \item $M$ je \textbf{popolno prirejanje}, če $\forall v \in V \exists e \in M : v \in e$\\
    (Vsako vozlišče ima natanko eno rdečo povezavo)\\\
\end{itemize}
Naj bo $G$ dvodelene graf, $V = X \cup Y$\\
\begin{itemize}
    \item $M$ je popolno prirejanje iz $X$ v $Y$, če je prirejanje in $\forall x \in X: \exists e \in M : x \in e$\\
    \item če obstaja popolno prirejanje iz $X$ v $Y$, je $|X| \leq |Y|$
\end{itemize}
Popolno prirejanje iz $X$ v $Y$ obstaja $\Leftrightarrow$ 
\[\forall A \subseteq X : |A| \leq |N(A)| \]
Če je $G$ biregularen graf, obstaja popolno prirejanje iz $X$ v $Y$ ali pa iz $Y$ v $X$.
\end{multicols}
\end{document}